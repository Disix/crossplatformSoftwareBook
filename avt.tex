\section*{Вступ}
Комп'ютерні технології застосовуються в численних предметних областях. Зокрема в засобах зв'язку, навігації, вбудованих в транспортні засоби системах управління, побутовій техніці, біомедичних приладах і т.д. Усі ці комп'ютерні системи є гетерогенними, тобто мають різну конфігурацію апаратного забезпечення та операційних систем. 

Наприклад, більшість мобільних пристроїв (телефони, планшети, букрідери, засоби GPS навігації) використовують доволі потужні центральні процесори, достатньо великі об'єми оперативної пам'яті, дисплеї великої роздільної здатності та ОС Android, iOS чи Windows (подано в алфавітному порядку). Тоді як маршрутизатори фірми Cisco використовують енергоекономічні, низькочастотні центральні процесори, малий об'єм оперативної пам'яті, не обладнані дисплеєм та працюють під керуванням Internet Operating System. 

Комбінацію апаратного забезпечення та операційної системи надалі будемо називати {\it платформою}. 

Використання апаратних ресурсів платформи вимагає розробки програмного забезпечення (ПЗ). Розробку нового ПЗ доречно виконувати на основі модулів повторного використання (reusable units): фреймворків, бібліотек класів, засобів розробки (SDK) та інтегрованих середовищ розробки (IDE).

Досвід, отриманий на одній платформі чи предметній області може бути використаний при розробці ПЗ для іншої платформи чи предметної області. Тому ПЗ та модулі повторного використання проектують в такий спосіб, щоб їх можна було без модифікацій зібрати (build) і використовувати на кількох платформах одночасно, що й називається {\it кросплатформністю}.

На думку автора, комплексна дисципліна ``Інженерія кросплатформного програмного забезпечення'' складається з трьох складових дисциплін: ``Конструювання ПЗ'', ``Технологія Java'' та ``Технологія .NET''.

Дисципліна ``Конструювання ПЗ'' стосується таких питань як
\begin{itemize} 
\item конвенції щодо оформлення коду на різних мовах програмування (відступи, дужки, іменування): для Java --- [\url{http://www.oracle.com/technetwork/java/codeconvtoc-136057.html}], 
для C\#.NET --- [\url{http://msdn.microsoft.com/en-us/library/ff926074.aspx}]
\item користування засобами автоматизованого документування коду на основі анотацій: javadoc [\url{http://www.oracle.com/technetwork/java/javase/documentation/index-137868.html}] для Java; XML Documentation [\url{https://www.simple-talk.com/dotnet/.net-tools/taming-sandcastle-a-.net-programmers-guide-to-documenting-your-code/}] та NDoc для .NET, 
\item теоретичні засоби мінімізації складності за допомогою грамотної інкапсуляції та дизайн паттернів (на основі фундаментальної книги Стівена Макконнелла про конструювання ПЗ),
\item засоби модульного (unit) тестування: JUnit для Java; NUnit чи MSUnit для .NET.
\item засоби автоматизації функціонального (інтеграційного) тестування, або, іншими словами, UI тестування, наприклад, сервіс \url{http://testfairy.com} та SDK засіб uiautomator для Java-Android [\url{http://developer.android.com/tools/help/uiautomator/index.html}].
\end{itemize}

Дисципліна ``Технологія Java'' описує основні питання платформи Java [\url{http://www.oracle.com/technetwork/topics/newtojava/java-technology-concept-map-150250.pdf}], а саме:
\begin{itemize}
\item особливості Java як мови програмування: структура проекту, пакети як аналог простору імен в С++ та С\#, інтерфейс Enumerable, узагальнені класи, успадкування, область видимості, анонімна реалізація інтерфейсів як аналог lambda операторів в С\#, порівняння об'єктів та значень, переозначення методів equals та clone, обчислювальні потоки чи нитки, засоби синхронізації на основі моніторів, ввід-вивід, користування командним рядком java, javac та змінними середовища CLASSPATH, JAVA\_HOME, ANDROID\_HOME,
\item засоби автоматизації збирання, тестування і публікування Ant та gradle, репозиторії бібліотек maven,
\item інтегроване середовище розробки InelliJ Idea чи Android Studio (засоби інспекції коду та рефакторингу),
\item основи розробки для J2SE (файловий ввід-вивід, серіалізація),
\item основи розробки для ОС Android: життєвий цикл Activity, робота з ресурсами (стрічки, зображення, звук), форматування layout, обробка подій, робота зі списками ListView, перенесення даних з однієї активності в іншу, особливості проектування застосунків для Андроід (синглтон застосунку).
\end{itemize}

В рамках дисципліни ``Технологія .NET'' будуть вивчатися:
\begin{itemize}
\item особливості мови С\#: структура solution, простори імен, lambda вирази, інтерфейс IEnumerable, події, делегати, робота з обчислювальними потоками (нитками), засоби синхронізації, робота з командним рядком і файловою системою
\item робота з Windows Forms: форматування UI, обробка подій, робота з графікою,
\item факультативно --- розробка кросплатформних (для ОС Android, iOS, Windows) мобільних додатків за допомогою Xamarin SDK і фреймворка MvvmCross на мові програмування С\# та бібліотек класів .NET.
\end{itemize}
